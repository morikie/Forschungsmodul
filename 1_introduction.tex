\section{Introduction}
Every year about 200 million people are contracted with Malaria. Half a million to one million people succumb to the disease annually, especially children and elderly persons. It is, therefore, along with AIDS and tuberculosis one of the most common infectious diseases worldwide. 

Malaria is predominantly found in tropical and subtropical regions. The high spread of Malaria in these regions can be attributed to mosquitos which function as vectors for a unicellular, eukaryotic organism of the genus Plasmodium. 

Plasmodia are the pathogens that cause Malaria. They are parasites that live in the human liver cells and blood cells where they primarily feed on hemoglobin, needed for replication. A more detailed description of the life cycle of Plasmodium can be found in Fig. \ref{plasmodium_life_cycle}.

There are five known species of the Plasmodium genus that infect humans: P. vivax, P. falciparum, P. malariae, P. ovale and P. knowlesi, with falciparum being responsible for the most deaths. 


\begin{figure}[ht!]
	\centering
	\centerline{\includegraphics[width=0.8\textwidth]{pictures/malaria-life-cycle.png}}
	\caption[Life cycle of Plasmodia]{Structure of a eukaryotic mature mRNA. From left to right: 5' cap, 5' UTR, coding sequence, 3' UTR and poly(A) tail. Source: \url{https://en.wikipedia.org/wiki/File:MRNA_structure.svg}}
	\label{plasmodium_life_cycle}
\end{figure}

