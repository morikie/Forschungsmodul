\section{Introduction}
Malaria is a disease that is responsible for half a million to one million human deaths every year. The pathogen is a unicellular, eukaryotic organism called Plasmodium that lives parasitically in the human liver and blood stream. There are five known species of the Plasmodium genus that infect humans: P. vivax, P. falciparum, P. malariae, P. ovale and P. knowlesi, with falciparum being responsible for the most deaths. 

Malaria is widespread in tropical and subtropical regions which can be found near the equator. The high spread of Malaria in these regions can be attributed to mosquitos which function as vectors for Plasmodium. Due to their ubiquitousness there are over 200 millions infected people each year