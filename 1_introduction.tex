\section{Introduction}
Every year about 200 million people are contracted with Malaria. Half a million to one million people succumb to the disease annually, especially children and elderly persons. It is, therefore, along with AIDS and tuberculosis one of the most common infectious diseases worldwide. 

Malaria is predominantly found in tropical and subtropical regions. The high spread of Malaria in these regions can be attributed to mosquitos which function as vectors for a unicellular, eukaryotic organism of the genus Plasmodium. 

Plasmodia are the pathogens that cause Malaria. They are parasites that live in the human liver cells and blood cells where they primarily feed on hemoglobin. A more detailed description of the life cycle of Plasmodium can be found in Fig. \ref{plasmodium_life_cycle}.

There are five known species of the Plasmodium genus that infect humans: P. vivax, P. falciparum, P. malariae, P. ovale and P. knowlesi, with P. falciparum being responsible for the most deaths. 

The containment of Malaria is usually implemented prophylactically by reducing the Mosquito population. 
The common types are:
\begin{itemize}
\item Chemical: insecticide (like the controversial DDT) or agents that target the larvae in the water;
\item Biological: introducing natural predators like the mosquitofish, certain dragonfly species or viruses and bacteria that target mosquitos;
\item Natural: destruction of breeding grounds (i.e. draining/removing water) or the usage of mosquito nets.
\end{itemize} 
In case of an infection several antimalarial drugs are available like artemisinin, DOX (doxycycline), chloroquine and many more. However, the increasing resistance of Plasmodia to certain drugs is a serious problem. Often a combination of drugs has to be used to treat a patient successfully. Hence, scientists search for new drugs and molecular starting points to specifically target the Plasmodium cells. 

In this report I will introduce four papers that deal with the topic of finding new approaches to treat Malaria. 

\begin{figure}[ht!]
	\centering
	\centerline{\includegraphics[width=0.8\textwidth]{pictures/malaria-life-cycle.png}}
	\caption[Life cycle of Plasmodia]{Structure of a eukaryotic mature mRNA. From left to right: 5' cap, 5' UTR, coding sequence, 3' UTR and poly(A) tail. Source: \url{https://en.wikipedia.org/wiki/File:MRNA_structure.svg}}
	\label{plasmodium_life_cycle}
\end{figure}

